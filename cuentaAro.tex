\section{La cuenta en el aro}

Sea un aro con una cuenta (bolita) que se puede mover libremente alrededor del aro, por otro lado el aro va a rotar de manera horizontal, el movimiento se traduce en el angulo que forma la bolita.

$$
m\ddot{x}=F
$$
dado que nos interesa el desplazamiento en función del angulo, por definición de aceleración tangencial:
$$
m\ddot{x}=F=mR \ddot{\varphi}
$$
\begin{figure}[ht]
    \centering
    \incfig{cuenta}
    \caption{Cuenta en el aro}
    \label{fig:cuenta-en-el-aro}
\end{figure}

\begin{figure}[ht]
    \centering
    \incfig{detalles}               
    \caption{Componentes}
    \label{fig:detalles}
\end{figure}

dado que la suma de los ángulos de un \textcolor{Green}{triangulo} deben sumar 180 grados, tenemos entonces:
$$
\pi = \frac{\pi}{2} + \theta + \frac{\pi}{2}-\varphi \implies \theta=\varphi
$$
a su vez, recordemos que $$
\cos\varphi=\frac{\text{ady}}{\text{hip}} \implies \text{ady} = m\rho\omega^2\cos\varphi
$$
de igual manera:
$$
\sin\theta=  \sin\varphi=\frac{\text{op}}{\text{hip}} \implies \text{op}=mg\sin\varphi 
$$
además $$
\rho = R\sin\varphi
$$
Ahora, dado que $F=mR \ddot{\varphi}$ tiene que ser igual a todas las fuerzas tangenciales tenemos que:
$$
\begin{aligned}
  mR \ddot{\varphi} &= -mg\sin\varphi+\cos\varphi m \rho \omega^2 - b \dot{\varphi} \\
		    &= -mg\sin\varphi+\cos\varphi m R \sin\varphi\omega^2 - b \dot{\varphi} \\
		    &= -b \dot{\varphi}+m\sin\varphi \left( -g + cos\varphi R \omega^2 \right)  \\
		    &= -b \dot{\varphi}+mg\sin\varphi \left( \frac{R\omega}{g}\cos\varphi - 1 \right) 	 \\ 
\end{aligned}
$$
Aproximando esta ecuación suponiendo que $mR\varphi \simeq 0$:
$$
\begin{aligned}
  0 &= -b\dot{\varphi}+mg\sin\varphi \left( \frac{R\omega^2}{g}\cos\varphi-1 \right)   \\
  \dot{\varphi} &= \frac{mg}{b}\sin\varphi \left( \frac{R\omega^2}{g}cos\varphi-1 \right)=f(\varphi)   \\ 
\end{aligned}
$$
cuales serian los puntos de equilibrio? Serán aquellos donde la bolita nunca se va a mover, como cero y $\pi$

Puntos de equilibrio:
$$
  \frac{mg}{b}\sin \varphi^* \left( \frac{R\omega^2}{g}\cos\varphi^{*}-1 \right)=0  
$$
Caso 1: $\sin\varphi^{*}=0 \implies \varphi_1^{*}=0 , \varphi_2^{*}=-\pi $

Caso 2: $\frac{R\omega^2}{g}\cos\varphi^{*}-1=0$

Definiendo  $$
\gamma=\frac{R\omega^2}{g} >0
$$
$$
\begin{aligned}
  \gamma\cos\varphi^{*}-1 &= 0 \\
  \gamma \cos\varphi^{*}&=1  \\ 
  \cos\varphi^{*}&=\frac{1}{\gamma}>0 \ (\cos x \in [-1,1]  ) \implies \frac{1}{\gamma}<1 , \ \therefore \gamma>1	\\
\end{aligned}
$$
si $\gamma>1 \implies$ $$\varphi_{3,4}^{*}=\cos^{-1} \left( \frac{1}{\gamma} \right) $$

\begin{tcolorbox}[colback=Black!4,colframe=White] 
\begin{nota}
  Si $\gamma>1$ estamos diciendo que $\gamma $ es lo suficientemente grande, donde $g,R$ son constantes que ya están fijas, pero $\omega^2$ es la velocidad de giro, por lo que si giramos suficientemente rápido aparecen 2 puntos de equilibrio.
  
\end{nota}
\end{tcolorbox}

Estabilidad de los puntos de equilibrio.
$$
\begin{aligned}
  f'(\varphi)&=\frac{mg}{b} \left[ \cos\varphi(\gamma\cos\varphi-1)-\gamma\sin^2\varphi \right]  
\end{aligned}
$$

$$
f'(0)=\frac{mg}{b} \left[ 1(\gamma-1)-0 \right] =\frac{mg}{b}(\gamma-1)<0 \text{ si } \gamma<1
$$
esto significa que $\varphi=0$ es estable si $\gamma<1$

por otro lado

$$
f'(\pi)=\frac{mg}{b} \left[ -1(-\gamma-1) \right] = \frac{mg}{b}(\gamma+1) > 0 
$$
por lo que $\varphi=-\pi$ siempre es inestable, lo que tiene mucho sentido pues corresponde a la bolita en el punto más alto y con cualquier perturbación se va a mover.

$$
\begin{aligned}
  f' \left( \cos^{-1} \left( \frac{1}{\gamma} \right)  \right)&=\frac{mg}{b} \left( \cos \left( \cos^{-1} \left( \frac{1}{\gamma} \right)  \right) \right) \left( \gamma\cos \left( \cos^{-1} \left( \frac{1}{\gamma} -1 \right)  \right)  \right) - \gamma\sin^2 \left( \cos^{-1} \left( \frac{1}{\gamma} \right)  \right)  \\
&= \frac{mg}{b} \left[ \frac{1}{\gamma} \left( \gamma \frac{1}{\gamma} \right) - \gamma\sin^2 \left( \cos^{-1} \left( \frac{1}{\gamma} \right)  \right)   \right] - \frac{mg}{b} \gamma \sin^2 \left( \cos^{-1} \left( \frac{1}{\gamma} \right)  \right)   \\ 
\end{aligned}
$$
\begin{tcolorbox}[colback=Black!4,colframe=White] 
\begin{nota}[¿Quien en el arco coseno de un angulo?]
Dada la aplicación arco coseno:
  $$
\operatorname{arccos} \left( \frac{1}{\gamma} \right)= \theta 
$$
eso implica que  $$
\frac{1}{\gamma}=\cos\theta = \frac{\operatorname{CA}}{\operatorname{H}}
$$
por lo que podemos construir un triangulo como se muestra en la figura 3, tal que la hipotenusa sea $\gamma$ y el cateto adyacente sea 1 y por pitagoras $h=\sqrt{\gamma^2-1}$, de esta manera tenemos que
$$
\sin\theta= \frac{\sqrt{\gamma^2-1}}{\gamma}
$$
solo si $\gamma^2-1>0 \implies \gamma>1$
\end{nota}
\end{tcolorbox}

\begin{figure}[ht]
    \centering
    \incfig[0.35]{triangulo}
    \caption{$\operatorname{arccos}$ de un ángulo}
    \label{fig:triangulo}
\end{figure}

Por lo que regresando a la ecuación anterior:
$$
\begin{aligned}
  -\frac{mg}{b}\gamma\sin^2 \left( \cos^{-1} \left( \frac{1}{\gamma} \right)  \right) &= -\frac{mg}{b}\gamma\sin^2(\theta) \\
  &= -\frac{mg}{b}\gamma(\sin\theta)^2 \\ 
  &= -\frac{mg}{b}\gamma \left( \frac{ \sqrt{\gamma^2 - 1}}{\gamma} \right)^2  \\
  &= -\frac{mg}{b}\frac{\gamma^2-1}{\gamma} \\
  &= \frac{mg}{b} \left( \frac{1-\gamma^2}{\gamma}  \right) \\ 
\end{aligned}
$$
notemos que $$
\left( \frac{1-\gamma^2}{\gamma} \right)<0 \text{ si } 1-\gamma^2<0 \text{ ie } \gamma>1 
$$
lo cual siempre es verdad. Por lo que $\varphi^{*}_{3,4}$ siempre son estables cuando existen. Dado que tenemos un punto de equilibrio que pasa de estable a inestable, y 2 puntos de equilibrio que siempre son estables entonces tenemos una bifurcación de tipo tridente.



