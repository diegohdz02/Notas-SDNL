\section{Retrato fase no lineal}

\begin{tcolorbox}[colback=Black!4, colframe=White, arc = 0mm]
\textbf{Caso lineal}
$$
\begin{aligned}
  \dot{\vec{x}}=A \vec{x} 
\end{aligned}
$$
donde $A$ es una matriz 2x2, y $\vec{x} = \begin{pmatrix} x_1(t) \\ x_2(t) \end{pmatrix} $, si $\operatorname{det} \neq 0 $ entonces $\vec{x}^*=\begin{pmatrix} 0 \\ 0 \end{pmatrix}. $  
\end{tcolorbox}

\begin{tcolorbox}[colback=Black!4, colframe=White, arc = 0mm]
\textbf{Caso no lineal}
 $$
\begin{aligned}
  \dot{x_1} &= f_1(x_1,x_2) \\ 
  \dot{x_2} &= f_2(x_1,x_2) \\ 
  \dot{\vec{x}}&=\vec{f}(\vec{x})
\end{aligned}
$$
con $$
 \vec{f}: \mathbb{R}^2 \to \mathbb{R}^2
$$
a lo que le llamamos campo vectorial.

  \textbf{Características}

  \begin{enumerate}
    \item  $\begin{pmatrix} 0 \\ 0 \end{pmatrix} $ no necesariamente es un punto de equilibrio.
      \item Hay más de un punto de equilibrio.
	\item Cada punto de equilibrio tiene diferente estabilidad.
	  \item Hay trayectorias en todo $\mathbb{R}^2$
  \end{enumerate} 
\end{tcolorbox}

  
\begin{ejemplo} Sea el sistema $$
\begin{aligned}
  \dot{x}_1 &= x_1+e^{-x_2} \\ 
  \dot{x_2} &= -x_2 \\ 
\end{aligned}
$$
entonces nuestro campo vectorial estará definido como se sigue $$
\dot{\vec{x}}= \begin{pmatrix} x_1+e^{-x_2} \\ -x_2 \end{pmatrix} = \vec{f}(\vec{x})
$$
Sus puntos de equilibrio estarán dados por
\begin{equation}
  \begin{pmatrix} 0 \\ 0 \end{pmatrix} =  \vec{f}(\vec{x}^*)=\begin{pmatrix} x_1^*+e^{{-x_2}^*} \\ -x_2^* \end{pmatrix} 
\end{equation}
por lo que $-x_2^*=0$, es decir $x_2^*=0$, además se sigue que
$$
\begin{aligned}
    x_1^*+e^{-x_2^*} &= 0 \\
    x_1^* + e^{-0} &= 0 \\ 
    x_1^*+1 &= 0 \\
    x_1 ^* &= -1 
\end{aligned}
$$
$$
\vec{x}^* = (-1,0)
$$

\begin{tcolorbox}[colback=Black!4,colframe=White]
\begin{definicion}[Ceroclinas] Curvas sobre las que $f_1=0$ o $f_2=0$
\end{definicion}
\end{tcolorbox}


1ra ceroclina: $f_2=0 \implies x_2=0$ 

2da ceroclina: $f_1=0$
$$
\begin{aligned}
  x_1 + e^{-x_2}&=0 \\
  e^{-x_2} &= -x_1 \\ 
  -x_2 &= \ln(-x_1) \\ 
  x_2 &= -\ln(-x_1) \\ 
\end{aligned}
$$
\begin{figure}[H]
 \centering
 \subfloat[Dirección de los vectores velocidad del ejemplo 1]{
   \label{ceroclin} 
 \includegraphics[width=0.7\textwidth]{ceroclin.pdf}} \vfill
 \subfloat[Trayectorias solución]{
 \includegraphics[width=0.5\textwidth]{rtfnl.pdf}}
\end{figure}
para hacer la figura \ref{ceroclin} tenemos que hacer el siguiente análisis:

\begin{enumerate}
  \item Cuando $x_1=0$ entonces para la primera coordenada $x_1+e^{-x_2}=e^{-x_2} > 0 \ \forall x_2 \in \mathbb{R}$ por lo tanto la dirección de la velocidad es a la derecha, es por eso que en el cuadrante I las flechas horizontales $(x_1)$ va a la derecha. Por otro lado para la segunda coordenada $-x_2$, notemos que no depende de $x_1$ por lo tanto cuando $x_2>0 ,\ -x_2<0$ por lo tanto las flechas verticales apuntan hacia abajo. Y con estos dos vectores su vector resultante será la dirección del flujo.
  \item Para $x_2<0$ y $x_1=0$ entonces $e^{-x_2}>0$ por lo tanto debe apuntar a la derecha y para $x_1<0$ entonces $x_2<0$ por lo tanto $-x_2>0$ por lo que debe apuntar hacia arriba.
  \item Si $x_2=0$ entonces $x_1+e^{0}=x_1+1>0 \iff x_1>-1 $ análogamente tenemos que $x_1+1<0 \iff x_1<-1$
  \item Por último, sobre $-ln(-x)$ la coordenada  $x_1+e^{-x_2}$ es cero, por lo que solo nos debemos de fijar que pasa sobre $-x_2$, por consiguiente para $x_2>0$ entonces $-x_2<0$ por lo que serán flechas que apunten hacia abajo, y para $x_2<0$ entonces $-x_2>0$ entonces serán flechas que apunten hacia arriba. Las demás áreas serán combinaciones de lo que pasa en las fronteras, así que solo debemos seguir el vector resultante que estas forman.             
\end{enumerate}


\end{ejemplo}

\subsection{Teorema de existencia y unicidad en 2 dimensiones}

\begin{tcolorbox}[colback=Black!4,colframe=White]
\begin{teorema}[Teorema de existencia y unicidad]
  Sea $\dot{\vec{x}}=\vec{f}(\vec{x})$, $\vec{x}(0)=\vec{x}(0)$ un problema de condiciones iniciales. Supongamos que $\vec{f}(\vec{x})$ es continua al igual que todas sus primeras derivadas, $$
  \frac{\partial f_i}{\partial x_i}, 
  $$ en una región abierta y conexa de $\mathbb{R}^{n}$, $D \subset \mathbb{R}^{n}$, $\vec{x}_0 \in D$, tal que el problema de condiciones iniciales tiene solución y es única en un subintervalo $t \in (-\tau,\tau),\ \tau>0$.
  
\end{teorema}
\end{tcolorbox}

\begin{figure}[ht]
    \centering
    \incfig{teun}
    \caption{No son posibles si el teorema de existencia se cumple: no pueden existir dos curvas solución ni regiones donde no existan trayectorias, pues no existirían condiciones iniciales en dicha región. }
    \label{fig:teun}
\end{figure}

\begin{tcolorbox}[colback=Black!4,colframe=White] 
\begin{nota}[Ciclo límite]
  En $\mathbb{R}^2$ si las soluciones están confinadas en el plano, decimos que hay un \textbf{ciclo límite}, como se ve en la figura \ref{vander}.
  
\end{nota}
\end{tcolorbox}

\begin{figure}[H]
  \centering
  \includegraphics[width=0.5\textwidth]{vanderPol.png}
  \caption{Ecuación de van der Pol. Ejemplo de ciclo límite.}
  \label{vander}                                                
\end{figure}
