\subsection{Oscilaciones de relajación y bifurcaciones}

\begin{equation*}
  T \simeq 2 \int_{a}^{b}   d{t} = 2 \int_{x=2}^{x=1}  {dt}   
\end{equation*}                  

necesitamos expresar $dt$ como una función de  $x$. Las soluciones son  $x=x(t)$, usando la regla de la cadena:
\begin{align*}
  dt &= \frac{dt}{dx} dx 
\end{align*}
con el \textbf{teorema de funciones inversas} podemos hacer lo siguiente:
\begin{equation} \label{eq0_270422} 
	\frac{dt}{dx} = \frac{1}{\frac{dx}{dt}} = \frac{1}{\dot{x}} 
\end{equation}
queremos el intervalo que está entre 2 y 1, y cuando la solución esta en este intervalo la solución esta pegada a la ceroclina, por lo que la trayectoria es casi la ceroclina es decir $y \simeq F(x)$
\begin{equation} \label{eq1_270422} 
	\begin{split}
		\dot{y} &= \frac{dy}{dt} \\
		 &= \frac{d}{dt} \left( F(x) \right) \\
		  &= \frac{dF}{dx} \frac{dx}{dt} \\
			&= (x^2-1) \dot{x} 
	\end{split}
\end{equation}
y del sistema de ecuaciones sabemos que \begin{equation}   \label{eq2_270422} 
  \dot{y} = -\frac{x}{\mu}
\end{equation}
igualando (\ref{eq1_270422}) y (\ref{eq2_270422}) 
\begin{equation}  \label{eq3_270422} 
  \dot{x} = -\frac{x}{\mu(x^2-1)}
\end{equation}
y sustituyendo (\ref{eq3_270422}) en (\ref{eq0_270422}) 
\begin{align*}
	\frac{dt}{dx} &= \frac{1}{\dot{x}} \\
	 &= \frac{\mu(x^2-1)}{-x} \\
	  &= -\mu \left( x-\frac{1}{x} \right)  
\end{align*}
de esta manera 
\begin{align*}
	T &\simeq \int_{x=2}^{x=1}  d{t} \\
	 &= 2 \int_{x=2}^{x=1} \frac{dt}{dx} d{x} \\
	  &= 2 \int_{x=2}^{x=1} -\mu \left( x-\frac{1}{x} \right)  d{x} \\
	   &= 2 \int_{1}^{2} \mu \left( x-\frac{1}{x} \right)  d{x} \\
	    &= 2\mu \left. \left( \frac{x^2}{2} - \ln(x) \right)   \right|_{1}^2         \\
	     &= 2\mu \left( 2-\ln(2)-\frac{1}{2}+ln(1) \right)  \\
	      &= 2\mu \left( \frac{3}{2} - \ln(2) \right) \\
	       &= \mu (3-2\ln(2)) 
\end{align*}
por lo tanto si $ \mu=50$ entonces $T = 50(3-2\ln(2)) = 80.7$
\section{Bifurcaciones en 2D}

\subsection{Bifurcación de nodo silla en 2D}

\textbf{Forma normal}
$$\begin{aligned}
		\dot{x} &= \mu-x^2 \\
	\dot{y} &= -y
  \end{aligned}$$

	Para $ \mu<0$ tenemos un ``fantasma`` como se ve en la figura (\ref{bns-l1}) que apachurra nuestras soluciones dado que hace que la velocidad vaya hacia a la izquierda y mientras se hace pequeño reduce la velocidad del sistema. Para $ \mu=0$ tenemos un punto de equilibrio en $(0,0)$, y para $ \mu>0$ tenemos un nodo silla y un nodo estable. 
\begin{figure}[H]
 \centering
  \subfloat[$ \mu<0$]{
   \label{bns-l1}
    \includegraphics[width=0.3\textwidth]{bns2d-0.pdf}}
		\subfloat[$ \mu=0$]{
		\includegraphics[width=0.3\textwidth]{bns2d-1.pdf}}
		\subfloat[$ \mu>0$]{
		\includegraphics[width=0.3\textwidth]{bns2d-2.pdf}}
		\caption{}
 \label{}
\end{figure}



