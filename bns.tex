\chapter{Bifurcaciones}  
\section{Bifurcación de nodo silla}
La forma normal del sistema está dada por la ecuación
\begin{equation*}
  \dot{x} = r + x^2 
\end{equation*}
con $r$ constante, $x=x(t)$. 
\begin{enumerate}[a)]
	\item $r<0$

    \textbf{Puntos de equilibrio.} Igualamos la ecuación a cero y resolvemos
\begin{align*}
r+x^{* 2} &=0 \\
x^{* 2} &=-r \\
\therefore x_{1,2}^{*} &=\pm \sqrt{-r}
\end{align*}
\textbf{Estabilidad de los puntos de equilibrio}

Derivamos la función con respecto de $x$ y evaluar la derivada con respecto a los puntos de equilibrio:
\begin{equation*}
\frac{d f}{d x}=2 x,\left.\quad \frac{d f}{d x}\right|_{x^{*}}=2 x^{*}
\end{equation*}
para la primera raíz tenemos
$$
\left.\frac{d f}{d x}\right|_{x_{1}^{*}}=2 \sqrt{-r}>0
$$
por lo tanto es inestable, análogamente para la segunda raíz
$$
\left.\frac{d f}{d x}\right|_{x_{2}^{*}}=-2 \sqrt{-r}<0
$$
por lo tanto es estable.

\item Si $r=0$
  $$
  \frac{d f}{d x}=2 x,\left.\frac{d f}{d x}\right|_{x^{*}}=0
  $$
  dado que es al mismo tiempo estable e inestable por ser idénticamente cero decimos que es un punto semi-estable.

  \item Si $r>0$
    
    No tiene puntos de equilibrio y dado que siempre es creciente las soluciones tienden a infinito.
\end{enumerate}

\begin{figure}[H]
 \centering
 \subfloat[$r<0$]{
 \includegraphics[width=0.3\textwidth]{nds-2.pdf}}
 \subfloat[$r=0$]{
 \includegraphics[width=0.3\textwidth]{nds-3.pdf}}
 \subfloat[$r>0$]{
 \includegraphics[width=0.3\textwidth]{nds-4.pdf}}
\end{figure}

\begin{tcolorbox}[colback=Black!4, colframe=White, arc=2mm]
\begin{definicion}[Gráfica de bifurcación]
  Gráfica donde se muestran los puntos de equilibrio y su estabilidad como función del parámetro. Representamos los puntos de equilibrio estables con líneas solidas y punteadas a los puntos inestables.
\end{definicion}
\end{tcolorbox}
\begin{figure}[H]
  \centering
  \includegraphics[width=0.6\textwidth]{nds-1.pdf}
  \caption{Gráfica de bifurcación para $\pm \sqrt{-r}$}
\end{figure}

\begin{tcolorbox}[colback=Black!4, colframe=White, arc=2mm]
\begin{nota}[Caracterización de la bifurcación de nodo silla] Un modelo de ecuaciones diferenciales tiene una bifurcación de nodo silla si a medida que sus puntos de equilibrio se juntan desaparecen.
\end{nota}
\end{tcolorbox}
\begin{tcolorbox}[colback=Black!4, colframe=White, arc=2mm]
\begin{definicion}[Punto crítico]
  Al punto donde se fusionan los puntos de equilibrio se llama punto crítico.
\end{definicion}
\end{tcolorbox}
\begin{ejemplo} Muestra que la ecuación $\dot{x}=r-x-e^{-x}$ tiene una bifurcación de nodo silla y encuentra el punto crítico donde ocurre la bifurcación.
\vspace{2mm}

  \textbf{Solución}

  Dado que la ecuación $r-x-e^{-x}$ es complicada de resolver, vamos a tomar las funciones
  \begin{align*}
    &F(x^*) = r-x^* \\
    &G(x^*)=e^{-x^*} 
  \end{align*}

  \begin{figure}[H]
   \centering
   \subfloat[$r \simeq 0.93$]{
   \includegraphics[width=0.45\textwidth]{nds-5.pdf}}
   \subfloat[$r=2$]{
   \includegraphics[width=0.45\textwidth]{nds-6.pdf}}
  \end{figure} 

  en $r=r_c, e^{-x}$ y $r-x$ son tangentes por lo que $F(x)=G(x)$ y además deben tener la misma pendiente por lo que  $F'(x)=G'(x)$ , de donde obtenemos las ecuaciones:
  \begin{align*}
    r-x &= e^{-x} \\
    -1 &= -e^{-x} \\
    1 &= e^{-x} \\
    \ln(1) &= \ln(e^{-x}) \\
    0 &= -x \\
    \therefore x&=0 
  \end{align*}
  regresando a que $r-x=e^{-x}$ entonces $r=1$.

  Para demostrar que tiene un punto de equilibrio nodo silla debemos llevar la ecuación a su forma normal, por lo que hacemos una expansión en series de taylor de la función exponencial:
  $$
\begin{aligned}
\dot{x} &=r-x-e^{-x} \\
&=r-x-\left(1-x+\frac{x^{2}}{2}+\ldots+\right) \\
&=(r-1)-\frac{x^{2}}{2}+\theta\left(x^{3}\right) \\
& \approx(r-1)-\frac{x^{2}}{2}
\end{aligned}  
  $$
  recordando que la bifurcación se da en $r=0$ entonces por la forma normal tenemos que $r-1=0$ por lo tanto $r=1$.
\end{ejemplo}

  
