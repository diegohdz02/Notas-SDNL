\documentclass[letterpaper]{article}
\usepackage[spanish]{babel}
\usepackage{geometry}
\geometry{top=1.5cm, bottom=2.5cm, left=2.5cm, right=2.5cm}
\usepackage{setspace}
\setlength{\parindent}{0cm}
\usepackage[utf8]{inputenc}
\usepackage[T1]{fontenc}
\usepackage{setspace} 
\usepackage{amsmath,amsfonts,amssymb,mathtools}
\usepackage{graphicx,float}
\newtheorem{ejemplo}{Ejemplo}
\newtheorem{teorema}{Teorema}
\newtheorem{corolario}{Corolario}
\newtheorem{preposicion}{Preposici\'on}
\newtheorem{lema}{Lema}
\newtheorem{definicion}{Definici\'on}
\newtheorem{nota}{Nota}
\newtheorem{notacion}{Notaci\'on}
\newtheorem{conclusion}{Conclusi\'on}
\usepackage[dvipsnames]{xcolor}
\usepackage{tcolorbox}
\usepackage{caption}
\usepackage{subcaption}

\title{Resumen: Bifurcaciones}
\author{Diego Eduardo Hernández Cerda\\
}

\begin{document}
\maketitle

\begin{tcolorbox}[colback=YellowGreen!10, colframe=YellowGreen, title=Tipos de bifurcaciones y características, center title, fonttitle=\bfseries]

  \tcbsubtitle{Bifurcación de Nodo-Silla}

  Forma normal. $$
  \dot{x}=r+x^2
  $$
  Puntos de equilibrio 

  Para $r<0$ hay dos puntos de equilibrio en  $\hat{x}_{1,2}= \pm \sqrt{-r}$, donde $-\sqrt{-r}$ es estable y $\sqrt{-r}$ es inestable.
  Para $r=0$ hay un punto de equilibrio en $r=0$ el cual es semiestable.
  
  Para $r>0$ no hay puntos de equilibrio.

  \tcbsubtitle{Bifurcación transcrítica}

  Forma normal.
  $$
  \dot{x}=rx-x^2  $$

  Puntos de equilibrio

  Siempre existe $\hat{x}_1=0$ como punto de equilibrio sin importar el valor de $r$, donde:
  \begin{enumerate}
    \item Si $r>0$ entonces es inestable.
      \item Si $r < 0$ entonces es estable
	\item Si $r = 0$ no se puede determinar que es lo que está pasando con el equilibrio, ya que los puntos se fusionan.
  \end{enumerate}

  Hay un punto de equilibrio en $\hat{x}_{2}=r$, donde:
  \begin{enumerate}
    \item Si $r>0$ entonces es estable.
      \item Si $r<0$ entonces es inestable.
	\item Si $r=0$ no se puede determinar la estabilidad. 
  \end{enumerate}

  \textbf{Este tipo de bifurcación se caracteriza por el cambio de estabilidad en sus puntos de equilibrio al encontrarse}      

  \tcbsubtitle{Bifurcación de tridente supercrítica}
  Forma normal.
  $$
  \dot{x} =rx-x^3
  $$
  Puntos de equilibrio.

  Tiene un punto de equilibrio en $\hat{x}_1=0$ que siempre existe, además de $$
  \hat{x}_{2,3} = \pm \sqrt{-r} \text{ si } r<0.
  $$
  Estabilidad.

  Si $r>0$ entonces $\hat{x}_1$ es inestable, si $r<0$ entonces $\hat{x}_1$ es estable.
  Por otro lado, dado que para que existan $\hat{x}_{2,3} \ r>0$ entonces $\hat{x}_{2,3}$ siempre son inestables. 
\end{tcolorbox}

\begin{tcolorbox}[colback=YellowGreen!10, colframe=YellowGreen, title=Tipos de bifurcaciones y sus características  ,fonttitle=\bfseries, center title]

\tcbsubtitle{Bifurcación de tridente subcrítica}

Forma normal. $$
\dot{x}=rx+x^3
$$
Puntos de equilibrio.

Siempre existe un punto de equilibrio  en $\hat{x}_1=0$, por otro lado $\hat{x}_{2,3}=\pm \sqrt{-r}$ si $r<0$.

Estabilidad.

Si  $r<0$ entonces $\hat{x}_1$ es estable, mientras que si $r>0$ entonces es inestable.

Para $\hat{x}_{2,3}$, dado que $r>0$ entonces siempre son inestables.

Para casi todas las aplicaciones la bifurcación tridente supercrítica tendrá la forma normal  $$
\dot{x} =rx+x^3-x^{5}
$$
\textbf{Su característica principal es que este tipo de bifurcaciones se dan en funciones simétricas.}

Para $r>0$ las soluciones siempre explotan (se van a infinito en tiempos finitos.)

\end{tcolorbox}

\end{document}

