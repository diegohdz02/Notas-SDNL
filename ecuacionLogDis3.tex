\section{Ecuación logística discreta 3}
\begin{equation*}
  x_{n+1}=rx_n(1-x_n) 
\end{equation*}
\begin{comment}
  las lineas blancas de la gráfica quieren decir 
  el diagrama de órbitas nos dicen que para un valor de $r$ cuales son los valores que alcanzan la solución ya después que t tiende a infinito
  tomando $r=3.7$ y en el eje  $x$ estamos graficando para tiempos muy grandes los valores que toman la solución, entonces 
  
  diagrama de órbitas, nosotros  simulamos x_n para n muy grande, para cada valor de r, con condiciones iniciales  aleatorias, graficamos un punto
  en el valor de, para x<1 solo veíamos una linea en hasta x=1, para 1 a 3 teníamos un punto de equilibrio estable, por lo que solo vemos una sola linea
  por que todas las condiciones iniciales convergían  al punto de equilibrio, para r=3 había un doblamiento de periodo, entonces ya no había punto de equilibrio estable
  dependiendo de la condición inicial  el a_n podía quedar en el punto de arriba o de abajo, en 3.4 había otro doblamiento de periodo, entonces el ciclo de periodo 2 inestable 
  se convertía en un ciclo de periodo 4.
  Los manchones blancos son soluciones caóticas y se les llaman ventanas periódicas , lo azul son soluciones periódicas.

  recuerda que las bifurcaciones en donde de la nada aparecen 2 puntos de equilibrio se le llama "nodo silla" 

  Soluciones intermitentes: si nos fijamos en cierto punto en n, parece que la solución tiene cierta periocidad, pero despues de cierto n hace cosas aleatorias -> periodicas -> aleatoriedad -> periocidad

  Gráfica tercer iterado para $r \simeq 3.79$ (3er iterado nos va a decir como se comporta la solución 3 puntos despues $x_n + 3$)


\end{comment}

Exponente de Lyapunov
Caos; 1) Sensibilidad de condiciones iniciales
$x_0$ 1a condición y $x_0+\delta_0$ segunda concidión tal que cada una resulta en una solución $x_n,x_n'$ respectivamente, tenemos que  $delta_n = \left| x_n - x_n' \right| $, supongamos que la separación entre las soluciones al tiempo en es exponencial, esto es
\begin{equation*}
  \left| \delta_n \right| \simeq \left| \delta_0 \right| e^{n \lambda}  
\end{equation*}
$\lambda$ se denomina exponente de Lyapunov.
\begin{align*}
  x_0 \to x_n & x_n = f^{n}(x_0) \\
  x_0+\delta_0 \to x_n' & x_n' = f'(x_0+\delta_0)
\end{align*}
\begin{equation*}
  \delta_n = \left| f^{n}(x_0)-f^{n}(x_0-\delta_0) \right|  = \left| f^{n}(x_0)-f^{n}(x_0+\delta_0) \right| 
\end{equation*}
\begin{equation*}
  \lambda = \frac{1}{n} \ln \left| \frac{\delta_n}{\delta_0} \right| = \frac{1}{n} \ln \left| \frac{f^{n}(x_0+\delta_0)-f^n (x_0)}{\delta_0} \right|  
\end{equation*}
tomando $\lim_{\delta_0 \to 0} $:
\begin{equation*}
  \lambda = \lim_{\delta_0 \to \infty} \frac{1}{n} \ln \left| \left( f^{n} \right)' (x_0) \right|  
\end{equation*}
donde
\begin{equation*}
  \begin{split}
    (f^{n})' &= [f(f(f(f)\cdots (f(x_0)))\cdots)]' \\
             &= f'(f(f\cdots(x_0))\cdots)[f(f \cdots f(x_0))] \\
             &= f'(f^{n-1}(x_0))[f^{n-1}] \\
             &= \prod_{i=0}^{n-1} f'(x_i) 
  \end{split}
\end{equation*}
Sustituyendo tenemos
\begin{equation*}
  \lambda = \frac{1}{n} \ln \left| \prod_{i=0}^{n-1} f'(x_i)   \right|  = \frac{1}{n} \sum_{i=1}^{n-1} \ln \left| f'(x_i) \right|   
\end{equation*}
si $\lambda>0$ tenemos soluciones caóticas, pues se separan exponencialmente con el tiempo, si $\lambda<0$ son soluciones no caótica

Ec log
\begin{equation*}
  x_{n+1}=rx_n(1-x_n) 
\end{equation*}

Universalidad: La secuencia de soluciones periódicas es la misma.

La proporción de valores del parámetro entre cada solución periódico (entre cada bifurcación de doblamiento de periodo es la misma) es decir, digamos que $r_n$ es el valor de  $r$ donde se da la  n-esima bifurcación de doblamiento de periodo, entonces la constante de Feigenbaum definida como
 \begin{equation*}
  \delta = \lim_{n \to \infty} \frac{r_n - r_{n-1}}{r_{n+1}-r_n} = 4.669 
\end{equation*}


