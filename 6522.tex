\section{Bifurcaciones globales}
Este tipo de bifurcaciones se caracteriza a diferencia de las bifurcaciones anteriores en que \textbf{no involucran puntos de equilibrio individuales que cambien de estabilidad, sino regiones amplias del espacio donde hay un cambio en el comportamiento de las soluciones}. 
\subsection{Bifurcación de silla-nodo de ciclos}
\begin{ejemplo}
  \begin{align*}
    \dot{r} &= \mu r + r^2- r^{5} \\
    \dot{\theta} &= \omega + br^2 
  \end{align*}
  Tomando $\dot{r}=\mu r + r^3-r^{5}$ lo podemos reescribir:
  \begin{align*}
    \dot{r} &= r (\mu + r^2 - r^{4})
  \end{align*}
  por lo que $r^*=0$ es un punto de equilibrio, además:
  \begin{align*}
    \mu+r^2-r^{4} &= 0  \quad  \left( s \equiv r^{{*}^2} \right)  \\
    \mu + s - s^2&= 0 \\              
    s^2-s-\mu &= 0
  \end{align*}
  utilizando la chicharronera:
  \begin{align*}
      s &= \frac{1\pm \sqrt{1+4\mu}}{2} \\
    	r^{{*}^2} &= \frac{1\pm \sqrt{1+4\mu}}{2} \\
    	r^* &= \sqrt{\frac{1\pm \sqrt{1+4\mu}}{2}}
  \end{align*}
  dado que $r^*$ es un radio, entonces tenemos únicamente 2 puntos de equilibrio:
  \begin{align*}
    	r^*_{1,2} &= \sqrt{\frac{1\pm \sqrt{1+4\mu}}{2}}
  \end{align*}
  por lo tanto tenemos 2 soluciones que se convierten en una sola cuando $r_1^*=r_2^*$ es decir:
  \begin{align*}
    1+4\mu_c &= 0 \\
    4\mu_c&= -1 \\
    \mu_c&= -\frac{1}{4}
  \end{align*}
  entonces
  \begin{equation*}
    r_c^*=\sqrt{\frac{1}{2}} = \frac{1}{\sqrt{2}} = \frac{\sqrt{2}}{2} \simeq 0.7071 
  \end{equation*}

  \begin{figure}[H]
   \centering
   \subfloat[$ \mu<\mu_c$]{
   \includegraphics[width=0.3\textwidth]{gb1.pdf}}
   \subfloat[$ \mu=\mu_c$]{
   \includegraphics[width=0.3\textwidth]{gb2.pdf}}
   \subfloat[$ \mu>\mu_c$]{
   \includegraphics[width=0.3\textwidth]{gb3.pdf}}
  \end{figure}
  las soluciones se ven como se sigue:
  \begin{figure}[H]
   \centering
   \subfloat[$ \mu<\mu_c$]{
   \includegraphics[width=0.3\textwidth]{gb4.pdf}}
   \subfloat[$ \mu = \mu_c$]{
   \includegraphics[width=0.3\textwidth]{gb5.pdf}}
    \subfloat[$ \mu>\mu_c$]{
      \includegraphics[width=0.3\textwidth]{gb6}}
  \end{figure}
  podemos observar que de $ \mu<\mu_c$ no existe ningún ciclo limite, para $\mu=\mu_c$ se crea de la nada un ciclo límite que no es estable ni inestable, y para $ \mu>\mu_c$ se divide en dos el ciclo límite, donde uno es estable y el otro es inestable.  
\end{ejemplo}

\subsection{Bifurcación de periodo infinito}

\begin{ejemplo}
  \begin{align*}
    \dot{r} &= r(1-r) \\
    \dot{\theta}&= \mu-\sin\theta
  \end{align*}
  Notemos que $\dot{r}=r(1-r)$ es la ecuación logística, y tiene un punto de equilibrio en $r^*=0$ inestable, mientras que en $r^*=1$ es un punto de equilibrio estable. Esto nos dice que siempre nos vamos a pegar a un circulo de radio 1.

  Por otro lado
  $\dot{\theta}=\mu-\sin\theta$ tiene un punto de equilibrio en $ \mu=\sin\theta^*$, si  $ \mu>1$ entonces no hay $\theta^*$, si $ \mu=1$ entonces $\theta^*=\frac{\pi}{2}$ eso quiere decir que en este punto nos vamos a detener, por lo que ya no vamos a girar, si $-1<\mu<1$ tenemos $\theta_1^*, \theta_2^*$ en los cuales ya no podemos seguir rotando. Esto gráficamente se ve como:

  \begin{figure}[H]
   \centering
    \subfloat[$ \mu>1$]{
      \includegraphics[width=0.3\textwidth]{gb7}}
    \subfloat[$ \mu=1$]{
      \includegraphics[width=0.3\textwidth]{gb8}}
    \subfloat[$ \mu>1$]{
      \includegraphics[width=0.3\textwidth]{gb9}}
  \end{figure}
  y las soluciones:
  \begin{figure}[H]
   \centering
    \subfloat[$ \mu>1$]{
      \includegraphics[width=0.3\textwidth]{gb10}}
    \subfloat[$ \mu=1$]{
      \includegraphics[width=0.3\textwidth]{gb11}}
    \subfloat[$mu>1$]{
      \includegraphics[width=0.3\textwidth]{gb12}}
  \end{figure}

  \begin{tcolorbox}[colback=Black!4, colframe=White,arc=2mm]
  \begin{recordatorio}[Velocidad angular]
    $\dot{\theta}$ representa la velocidad angular (que tan rápido recorro el angulo). 
  \end{recordatorio}
  \end{tcolorbox}
\end{ejemplo}
si tomamos una solución particular para cada caso se verían de la siguiente forma:
\begin{figure}[H]
 \centering
  \subfloat[Tenemos una solución oscilatoria puesto que el periodo aproximadamente es de $T \simeq 10$]{
    \includegraphics[width=0.3\textwidth]{gb13}}
  \subfloat[$T = \infty$]{
    \includegraphics[width=0.3\textwidth]{gb14}}
    \subfloat[La solución no oscila]{
    \includegraphics[width=0.3\textwidth]{gb15}}
\end{figure}

\subsection{Bifurcación homoclínica}
En estos casos será muy complicado llegar a resolver las ecuaciones de manera analítica, para poder obtener los puntos de bifurcación, por lo tanto estas bifurcaciones las abordaremos de manera computacional.
\begin{ejemplo}
  \begin{equation}
    \label{bh1} 
    \begin{split}
      \dot{x} &= y \\
      \dot{y} &= \mu y + x - x^2 + xy 
    \end{split}
  \end{equation}
  \begin{figure}[H]
   \centering
    \subfloat[]{
     \label{}
      \includegraphics[width=0.45\textwidth]{gb16}}
    \subfloat[]{
     \label{bh3}
      \includegraphics[width=0.45\textwidth]{gb17}}
      \caption{Diagrama de flujo del sistema (\ref{bh1}) para $ \mu< \mu_c$ }
   \label{}
  \end{figure} 
  del sistema (\ref{bh1}) podemos observar que $(0,0)$ y $(0,1)$ son puntos de equilibrio; donde $(0,0)$ es un punto silla y  $(0,1)$ es un punto inestable. A primera instancia podemos notar algunos detalles de la gráfica (\ref{bh3}); en primer lugar las flechas de color rojo (variedad inestable del punto silla) tienden a un ciclo límite, y la variedad estable (flechas negras) nos llevan al punto de equilibrio.
  \begin{figure}[H]   
    \centering
    \includegraphics[width=0.6\textwidth]{gb18}
    \caption{Solución del sistema con $ \mu<\mu_c$ con condiciones iniciales $(0.98,0.01)$} 
    \label{bh4}
  \end{figure}
  En la gráfica (\ref{bh4}) podemos ver el comportamiento de la solución rosa de la gráfica (\ref{bh3}) la cual oscila en el ciclo límite.
  \begin{figure}[H]
   \centering
    \subfloat[A la trayectoria roja le llamaremos variedad homoclínica, dado que conecta a un punto consigo mismo.]{
     \label{}
      \includegraphics[width=0.4\textwidth]{gb19}} 
    \subfloat[Solución con condiciones iniciales $(0.98,0.01)$]{
     \label{bh5}
      \includegraphics[width=0.49\textwidth]{gb20}}
      \caption{$ \mu=\mu_c$}
  \end{figure} 
  En la figura (\ref{bh5}) podemos apreciar como a la solución le cuesta cada vez más llegar al mismo punto, el periodo de oscilación se vuelve infinito para oscilaciones posteriores. Mientras que en la figura (\ref{bh6}) podemos apreciar que si partimos muy cerca del punto inestable, vamos a irnos a menos infinito. 
  \begin{tcolorbox}[colback=Black!4, colframe=White,arc=2mm]
  \begin{definicion}[Variedad homoclínica]
    Trayectoria que conecta un punto consigo mismo.
  \end{definicion}
  \end{tcolorbox}
  \begin{figure}[H]
   \centering
    \subfloat[]{
      \includegraphics[width=0.4\textwidth]{gb21}} 
    \subfloat[Solución con condiciones iniciales $(0.98,0.01)$]{
      \includegraphics[width=0.49\textwidth]{gb22}}
      \caption{$ \mu<\mu_c$}
      \label{bh6}
  \end{figure} 
  
\end{ejemplo}

