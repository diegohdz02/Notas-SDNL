
\section{Bifurcación transcrítica}

El sistema más sencillo que tiene esta bifurcación es su forma normal:
  $$
\dot{x} = rx-x^2
$$
\subsection*{Análisis de estabilidad lineal}

1.Primero tenemos que encontrar los puntos de equilibrio constantes reales:
$$
rx-x^2=0 \implies x(r-x)=0
$$
por lo que tenemos un punto de equilibrio en $\hat{x}_1=0$ que no depende de los parámetros (su valor será el mismo sin importar el valor de $r$)
por otro lado $\hat{x}_2=r$, y cuando $ r=0 $ entonces  $\hat{x}_1=\hat{x}_2$.

2.Calculamos la estabilidad de los puntos de equilibrio
$$ \frac{df}{dx}=r-2x $$

Para $\hat{x}_1$ :

$$  \left. \frac{df}{dx}  \right|_{x=0}=r-2(0)=r  $$
si $r>0$ entonces $$
\left. \frac{df}{dx} \right|_{x=0}>0 
$$
por lo tanto es un punto de equilibrio inestable, mientras que si $r<0$ entonces
 $$
\left. \frac{df}{dx} \right|_{x=0}<0
$$
por lo que es estable.

\begin{tcolorbox}[colback=Black!4, colframe=White, arc=2mm]
\begin{nota}
  Si $r=0$ no se puede determinar que es lo que está pasando, los puntos de equilibrio se fusionan y pasan cosas raras en cuanto a la estabilidad. Cuando un punto de equilibrio es cero es indicación de que hay una bifurcación en ese punto.

\end{nota}
\end{tcolorbox}

Para $\hat{x}_2$ :
$$
\left. \frac{df}{dx} \right|_{x=r} = r-2r=-r
$$
por lo tanto para $r>0$ entonces  $-r<0$ por lo tanto es estable, y cuando  $r<0$ entonces $-r>0$ por lo tanto es inestable (ver figura 2).
\vspace{2mm}
\begin{tcolorbox}[colback=Black!4, colframe=White, arc=2mm]
\begin{nota}[Caracterización de la bifurcación transcritica]
  Este tipo de bifurcación se caracteriza por el cambio de estabilidad en sus puntos de equilibrio al encontrarse.
\end{nota}
\end{tcolorbox}
\vspace{2mm}

\begin{figure}[H]
 \centering
  \subfloat[$r<0$]{
    \includegraphics[width=0.3\textwidth]{/Users/diego/Documents/NotasFac/SNL/imagenes/bf<0.pdf}}
  \subfloat[$r=0$]{
    \includegraphics[width=0.3\textwidth]{/Users/diego/Documents/NotasFac/SNL/imagenes/bf=0.pdf}}
  \subfloat[ $r>0$]{
    \includegraphics[width=0.3\textwidth]{/Users/diego/Documents/NotasFac/SNL/imagenes/bf>0.pdf}}
    \caption{Diagrama de flujo del sistema $\dot{x}=rx-x^2$}
\end{figure}

\begin{figure}[!tbp]
  \centering
  \includegraphics[width=0.5\textwidth]{/Users/diego/Documents/NotasFac/SNL/imagenes/bifgr.pdf}
  \caption{Diagrama de bifurcación del sistema $\dot{x}=rx-x^2$.}
\end{figure}   

\begin{lstlisting}[language=Mathematica]
biftn = Show[
  Plot[0, {r, -3, 0}, AxesLabel -> {r, SuperStar[x]}, 
   PlotStyle -> {Red, Thick}, PlotRange -> {{-3, 3}, {-3, 3}}, 
   AspectRatio -> 1], 
  Plot[0, {r, 0, 3}, AxesLabel -> {r, SuperStar[x]}, 
   PlotStyle -> {Red, Dashed}, PlotRange -> {{-3, 3}, {-3, 3}}, 
   AspectRatio -> 1], 
  Plot[r, {r, -3, 0}, AxesLabel -> {r, SuperStar[x]}, 
   PlotStyle -> {Blue, Dashed}, PlotRange -> {{-3, 3}, {-3, 3}}, 
   AspectRatio -> 1], 
  Plot[r, {r, 0, 3}, AxesLabel -> {r, SuperStar[x]}, 
   PlotStyle -> {Blue, Thick}, PlotRange -> {{-3, 3}, {-3, 3}}, 
   AspectRatio -> 1]]
\end{lstlisting}  

\begin{ejemplo}
  Probar que $ \dot{x}=x-a(1-e^{-bx}) $ presenta una bifurcación transcritica
\end{ejemplo}

Si $x=0$ entonces tenemos:  $$
0-a(1-e^{0})=0-a(1-1)=0 \implies f(0)=0 \quad \forall a,b \in \mathbb{R}
$$
es decir que $\hat{x}=0$ siempre es punto de equilibrio. \\

Dado que la ecuación no es lineal y además es del tipo trascendental no es tan fácil probar a mano que tiene este tipo de bifurcación,pero localmente de la bifurcación todas las ecuaciones se parecen a la forma normal, por lo que vamos a expandir $f(x)$ en series de Taylor alrededor de cero:
$$
f( \hat{x} )=0, \frac{df}{dx}=1-a(be^{-bx}), \left. \frac{df}{dx} \right|_{\hat{x}=0} =1-ab
$$
por lo que
$$
\frac{d^2f}{dx^2}=-a(-b^2e^{-bx}), \left. \frac{d^2f}{dx^2} \right|_{\hat{x}=0} = ab^2
$$
  \begin{tcolorbox}[colback=Black!4, colframe=White, arc=2mm]
  \begin{recordatorio}[Serie de Taylor]
    Una serie de Taylor es una aproximación de funciones mediante una serie de potencias alrededor de un numero real o complejo $a$ es la siguiente serie de potencias:
$$
\begin{aligned}
  \label{eq:}
  f(x) &\simeq f(a)+\frac{f'(a)}{1!}(x-a)+\frac{f''(a)}{2!}(x-a)^2 + \cdots + \frac{f^{(n)}(a)}{n!}(x-a)^{n}+ \cdots \\ &= \sum_{n=0}^{\infty} \frac{f^{(n)}(a)}{n!}(x-a)^{n}
\end{aligned}
$$   
  \end{recordatorio}
\end{tcolorbox}

De esta manera

\begin{equation}
  f(x) = \dot{x}  \simeq x(1-ab)+\frac{ab^2}{2}x^2
\end{equation}

y como siguiente paso vamos a acercar esta aproximación a su forma normal. 

Haciendo un cambio de variable
$$
y=\frac{ab^2}{2}x
$$
$$
\dot{y}=-\frac{ab^2}{2}x 
$$
multiplicando la ecuación (1) por $ \dot{y}$:

$$
-\frac{ab^2}{2} \dot{x}=-\frac{ab^2}{2}x(1-ab) - \left( \frac{ab^2}{2} \right) ^2x^2
$$
$$
\dot{y}=y(1-ab)-y^2 
$$
definiendo $R=(1-ab)$
entonces
 \begin{equation}
  \dot{y}=Ry-y^2 
\end{equation}

\begin{tcolorbox}[colback=Black!4, colframe=White, arc=2mm]
\begin{nota}
Dado que ya llegamos a la forma normal, esta es una prueba suficiente para mostrar que tiene una bifurcación transcritica. 
\end{nota}
\end{tcolorbox}

Recordando que la bifurcación ocurre cuando $R=0$ entonces  $(1-ab)=0$ es decir $ab=1$. \\

Análisis de estabilidad con $\hat{x}=0$:

$$
\left. \frac{df}{dx} \right|_{\hat{x}=0} = 1-ab
$$
Si $ab>1$ entonces $$
\left. \frac{df}{dx} \right|_{\hat{x}=0} <0
$$
por lo tanto es estable.
Si $ab<1$ entonces
 $$
\left. \frac{df}{dx} \right|_{\hat{x}=0} >0, 
$$
por lo tanto es inestable.

\begin{figure}[htpb]
  \centering
  \includegraphics[width=0.8\textwidth]{/Users/diego/Documents/NotasFac/SNL/imagenes/db1.pdf}
  \caption{Diagrama de bifurcación ejemplo 1}
\end{figure}

\begin{lstlisting}[language=Mathematica]
Manipulate[
 Plot[x - a*(1 - Exp[-b*x]), {x, -4, 4}, 
  PlotRange -> {{-4, 4}, {-1, 4}}, 
  AxesLabel -> {x, OverDot[x]}], {a, -3.5, 3.5}, {b, -2/3, 2/3}]
\end{lstlisting}  

\begin{lstlisting}[language=Mathematica]
ststex1ust = 
  ParallelTable[{a, 
    x /. 
     FindRoot[x - a*(1 - Exp[2*x/3]) == 0, {x, -4}]}, {a, -3.5, -3/2, 
    0.01}];

ststex1st = 
  ParallelTable[{a, 
    x /. FindRoot[x - a*(1 - Exp[2*x/3]) == 0, {x, 4}]}, {a, -3/2, -1/
     3, 0.01}];

ex1bdiag = 
 Show[ListPlot[ststex1ust, Joined -> True, 
   PlotStyle -> {Blue, Dashed}, AxesLabel -> {a, SuperStar[x]}, 
   PlotRange -> {{-3.5, -1/3}, {-4, 4}}], 
  ListPlot[ststex1st, Joined -> True, PlotStyle -> {Blue, Thick}, 
   AxesLabel -> {a, SuperStar[x]}, 
   PlotRange -> {{-3.5, -1/3}, {-4, 4}}], 
  Plot[0, {a, -3.5, -3/2}, PlotStyle -> {Red, Thick}, 
   AxesLabel -> {a, SuperStar[x]}, 
   PlotRange -> {{-3.5, -3/2}, {-4, 4}}], 
  Plot[0, {a, -3/2, -1/3}, PlotStyle -> {Red, Dashed}, 
   AxesLabel -> {a, SuperStar[x]}, 
   PlotRange -> {{-3.5, -1/3}, {-4, 4}}]]
\end{lstlisting}  

\begin{ejemplo}
  Probar que $ \dot{x}=r\ln x+x-1=f(x)$
\end{ejemplo}
Notemos que $f(1)=r\ln 1+1-1=0$, por lo que $\hat{x}=1$ es un punto de equilibrio para toda $r$.

Haciendo una expansión en series de Taylor alrededor de 1:
$$
\frac{df}{dx}=\frac{r}{x}+1, \ \left. \frac{df}{dx} \right|_{\hat{x}=1} = r+1, \ \frac{d^2f}{dx^2}=-\frac{r}{x^2}, \ \left. \frac{d^2f}{dx^2} \right|_{\hat{x}=1} =-r
$$
por lo que:
\begin{equation}
 f(x) \simeq (x-1)(r+1)+\frac{(x-1)^2}{2}(-r) = (x-1)(r+1)-\frac{r}{2}(x-1)^2
\end{equation}
haciendo un cambio de variable:
$$
\begin{aligned}
  y &= \frac{r}{2}(x-1) \\
  \dot{y} &= \frac{r}{2}  \\ 
\end{aligned}
$$multiplicando $ \dot{x}  $ por $\frac{r}{2}$ : 
$$
\begin{aligned}
  \frac{r}{2} \dot{x} &= \frac{r}{2}(x-1)(r+1)- \left( \frac{r}{2} \right) ^2 (x-1)^2  \\
  \dot{y} &= y(r+1)-y^2 \quad (R=r+1) \\
  \dot{y} &= Ry-y^2 \\ 
\end{aligned}
$$ con esta forma normal el punto de bifurcación se va a dar en $R=0$, es decir  $r=-1$. \\

Análisis de estabilidad para el punto $\hat{x}=1$ :
$$
\left. \frac{df}{dx} \right|_{\hat{x}=1} = r+1 
$$
Si $r>-1$ entonces
$$
\left. \frac{df}{dx} \right|_{\hat{x}=1} >0, \
$$
por lo tanto $\hat{x}=1$ es inestable. \\

Si $r<-1$ entonces
 $$
\left. \frac{df}{dx} \right|_{\hat{x}=1} < 0, \
$$
por lo tanto $\hat{x}=1$ es estable.

\begin{figure}[htpb]
  \centering
  \includegraphics[width=0.8\textwidth]{/Users/diego/Documents/NotasFac/SNL/imagenes/bf2.pdf}
  \caption{Diagrama de flujo ejemplo 2}
\end{figure}

\begin{lstlisting}[language=Mathematica]
Manipulate[
 Plot[r*Log[x] + x - 1, {x, 0, 2}, PlotRange -> {{0, 2}, {-1, 1}}, 
  AxesLabel -> {x, OverDot[x]}], {r, -2, 0}]
\end{lstlisting}  

\begin{lstlisting}[language=Mathematica]
ststex2ust = 
  ParallelTable[{r, 
    x /. FindRoot[r*Log[x] + x - 1 == 0, {x, 2}]}, {r, -1.5, -1, 
    0.01}];

ststex2st = 
  ParallelTable[{r, 
    x /. FindRoot[r*Log[x] + x - 1 == 0, {x, 0.1}]}, {r, -1, -1/2, 
    0.01}];

ex2bdiag = 
 Show[ListPlot[ststex2ust, Joined -> True, 
   PlotStyle -> {Blue, Dashed}, AxesLabel -> {r, SuperStar[x]}, 
   PlotRange -> {{-1.5, -1/2}, {0, 2.5}}], 
  ListPlot[ststex2st, Joined -> True, PlotStyle -> {Blue, Thick}, 
   AxesLabel -> {r, SuperStar[x]}, 
   PlotRange -> {{-1.5, -1/2}, {0, 2.5}}], 
  Plot[1, {r, -1.5, -1}, PlotStyle -> {Red, Thick}, 
   AxesLabel -> {r, SuperStar[x]}, 
   PlotRange -> {{-1.5, -1/2}, {0, 2.5}}], 
  Plot[1, {r, -1, -1/2}, PlotStyle -> {Red, Dashed}, 
   AxesLabel -> {r, SuperStar[x]}, 
   PlotRange -> {{-1.5, -1/2}, {0, 2.5}}]]
\end{lstlisting}  


