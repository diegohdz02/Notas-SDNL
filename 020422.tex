
\section{Bifurcación de Hopf} 

\begin{tcolorbox}[colback=Black!5, colframe=White, arc=2mm]
\begin{recordatorio}Tipos de bifurcaciones:
	
	Nodo silla
	\begin{align*}
	  \dot{x}&= \mu-x^2 \\
	  \dot{y} &= -y 
	\end{align*}
	Transcrítica
	\begin{align*}
	  \dot{x} &= \mu x -x^2 \\
	  \dot{y} &= -y 
	\end{align*}
	Tridente supercrítica
	\begin{align*}
		\dot{x}&=\mu x - x^3 \\ 
		\dot{y} &= -y \\ 
	\end{align*}
	Tridente subcrítica
	\begin{align*}
	  \dot{x} &= \mu x + x^3 \\
	  \dot{y} &= -y
	\end{align*}
	El punto de equilibrio siempre es $x=0$ cuando  $ \mu=0$.

	Analisis de estabilidad para conocer los eigenvalores del punto de equilibrio $x=0$ en el punto de bifurcación $ \mu=0$:
	\begin{gather*}
		J = \begin{pmatrix} -2x & 0 \\ 0 & -1 \end{pmatrix}  \begin{pmatrix} \mu - 2x & 0 \\ 0 & -1 \end{pmatrix}  \begin{pmatrix} \mu-3x^2 & 0 \\ 0 & -1  \end{pmatrix}  \begin{pmatrix} \mu+3x^2 &  0 \\ 0 & -1 \end{pmatrix} \\ 
		\left. J \right|_{\mu=0 , x = 0} = \begin{pmatrix} 0 & 0 \\ 0 & -1 \end{pmatrix}  \begin{pmatrix} 0 & 0 \\ 0 & -1 \end{pmatrix}  \begin{pmatrix} 0 & 0 \\ 0 & -1 \end{pmatrix}  \begin{pmatrix} 0 & 0 \\ 0 & -1 \end{pmatrix}  
	\end{gather*}
	tenemos entonces que $\lambda_1 = 0, \lambda_2=-1$. 
	A todas las bifurcaciones se le llama \textbf{bifurcación de eigenvalor cero.}
\end{recordatorio}
\end{tcolorbox}
\begin{tcolorbox}[colback=Black!5, colframe=White, arc=2mm]
	\begin{teorema}[Bifurcación de Hopf]
	Si tenemos un sistema bidimensional 
	\begin{align*}
	  \dot{x} &= f(x,y) \\
	  \dot{y} &= g(x,y) 
	\end{align*}
	que depende de un parámetro $ \mu$ y cuando $ \mu=\mu_0$ hay una bifurcación tal que los eigenvalores del punto de equilibrio dados por \begin{equation*}
	  \lambda(\mu) = \alpha(\mu) + i \beta (\mu)
	\end{equation*} 
	con $\alpha(\mu), \beta(\mu)\in \mathbb{R}$ y
	\begin{enumerate}
		\item $\alpha(\mu_0)=0, \beta(\mu_0)=\omega \neq 0$
		\item $ \left. \dfrac{d}{d \mu} \alpha(\mu)   \right|_{\mu=\mu_0}\neq 0 $
		\item $ \left. \Omega \right|_{\mu_0,x_0,y_0}  \neq 0$ con 
			\begin{equation*}
		  \begin{split}
				\Omega =& \frac{1}{16} \left( \partial_{x x x}f + \partial_{x y y}f + \partial_{x x y}g + \partial_{y y y}g \right)  + \frac{1}{16 \omega} ( \partial_{xy}f(\partial_{x x}f+\partial_{yy}f) \\
								&- \partial_{xy}(\partial_{x x}g+ \partial_{y y}g)  -\partial_{x x}f \partial_{x x}g + \partial_{yy}f \partial_{y y}g) 
		  \end{split}
		\end{equation*}
	\end{enumerate}
	entonces hay una bifurcación de Hopf supercrítica o subcrítica.
\end{teorema}
\end{tcolorbox}
La definición anterior lo que nos está diciendo es que si en el punto de bifurcación los eigenvalores son complejos tales que la parte real es cero y la parte imaginaria no lo es y además cruzan el eje imaginario con velocidad diferente de cero más la condición 3, entonces hay una bifurcación de Hopf. De esta manera tenemos que las bifurcaciones de Hopf son aquellas tales que al mover un parámetro los eigenvalores cruzan el eje imaginario sin pasar por el punto $(0,0)$ (pierden estabilidad sin pasar por el cero).   
\subsection{Bifurcación de Hopf supercrítica}
Forma normal
\begin{align*}
  \dot{x} &= \mu x - y - x(x^2+y^2) \\
  \dot{y} &= x + \mu y - y(x^2+y^2)  
\end{align*}
donde $(0,0)$ es un punto de equilibrio.
\vspace{2mm}

Matriz Jacobiana

\begin{equation*}
	J = \left. \begin{pmatrix} \mu - (x^2+y^2) - x(2x) & -1-x(2y) \\ 1-y(2x) & \mu-(x^2+y^2) - y (2y) \end{pmatrix} 
		\right|_{(0,0)} = \begin{pmatrix} \mu & -1 \\ 1 & \mu \end{pmatrix}  
\end{equation*}
de esta manera
\begin{equation*}
  \operatorname{tr}J = 2\mu
\end{equation*}
\begin{equation*}
  \operatorname{det} J = \mu^2+1
\end{equation*}
\begin{align*}
	\lambda_{1,2} &= \frac{\operatorname{tr}J \pm \sqrt{\operatorname{tr}J^2 - 4 \operatorname{det} J} }{2} \\
	 &= \frac{2\mu \pm \sqrt{4 \mu^2 - 4 (\mu^2+1)}}{2} \\
	  &= \frac{2 \mu \pm \sqrt{4a^2-4\mu^2-4}}{2} \\
	   &= \frac{2\mu \pm 2i }{2} \\
	    &= \mu \pm i 
\end{align*}
de esta manera tenemos que $$\operatorname{Re}\left(\lambda \right)=\mu,\ \operatorname{Im}\left(\lambda \right) = \pm i $$
por lo que si $ \mu<0$ entonces  $(0,0)$ es estable, por otro lado si $ \mu=0$ entonces $\lambda_{1,2}=\pm i$ y si $ \mu>0$ entonces $(0,0)$ es inestable y forma un ciclo límite estable.

\begin{figure}[H]
 \centering
  \subfloat[$ \mu<0$]{
   \label{bhs1}
    \includegraphics[width=0.3\textwidth]{1020522fig.pdf}}
  \subfloat[$ \mu=0$]{
   \label{bhs2}
    \includegraphics[width=0.3\textwidth]{2020522fig.pdf}}
  \subfloat[$\mu>0$]{
   \label{bh3}
    \includegraphics[width=0.3\textwidth]{3020522fig.pdf}}
\end{figure}

En el caso del diagrama de bifurcación lo que se hace es hacer una gráfica en tres dimensiones:

\begin{figure}[H]
	\centering
	\includegraphics[width=0.5\textwidth]{300522b1.pdf}
	\caption{Diagrama de bifurcación de Hopf supercrítica}
\end{figure}

\subsection{Bifurcación de Hopf subcrítico}
Forma normal
\begin{align*}
  \dot{x} &= \mu x - y + x(x^2 + y^2) \\
  \dot{y} &= x + \mu y + y(x^2+y^2) 
\end{align*}
si $\mu < 0$ tenemos un ciclo límite inestable y el $(0,0)$ es estable, si  $ \mu >0$ tenemos un punto de equilibrio inestable.
\begin{figure}[H]
 \centering
  \subfloat[$ \mu<0$]{
    \includegraphics[width=0.3\textwidth]{4020522fig.pdf}}
		\subfloat[$\mu=0$]{
		\includegraphics[width=0.3\textwidth]{5020522fig.pdf}}
		\subfloat[$ \mu>0$]{
		\includegraphics[width=0.3\textwidth]{6020522fig.pdf}}
\end{figure}
analogamente al caso anterior para el diagrama de bifurcación:
\begin{figure}[H]
	\centering
	\includegraphics[width=0.5\textwidth]{252-1}
	\caption{Diagrama de bifurcación de Hopf degenerado}
\end{figure}
\subsection{Bifurcación de Hopf degenerada}
En esta bifurcación ocurre lo mismo que en las anteriores, en el punto de equilibrio tienen eigenvalores que desestabilizan al punto, pero no cumplen alguna de las 3 condiciones anteriores. 
\begin{ejemplo}
	El siguiente sistema no cumple la 3ra condición del teorema:
	\begin{align*}
	  \dot{x} &= y \\
	  \dot{y} &= -\mu y - \sin x 
	\end{align*}
	donde  \begin{equation*}
	  \lambda_{1,2} = \frac{-\mu \pm \sqrt{\mu^2 -4}}{2}
	\end{equation*}

	\begin{figure}[H]
	 \centering
	  \subfloat[$\mu<0$]{
	    \includegraphics[width=0.3\textwidth]{3522-1}}
	  \subfloat[$\mu=0$]{
	    \includegraphics[width=0.3\textwidth]{3522-2}}
	  \subfloat[$\mu>0$]{
	    \includegraphics[width=0.3\textwidth]{3522-3}}
	\end{figure}
\end{ejemplo}


\begin{ejemplo}
	\begin{align*}
	  \dot{x} &= \mu x - y - xy^2 \\
	  \dot{y} &= x + \mu y + y^3 
	\end{align*}
	tarea moral: mostrar que el sistema se puede escribir como: 
	\begin{equation*}
	  \dot{r} = \mu r + r y^2 \ge \mu r 
	\end{equation*}
	por lo que $r(t)=r_0e^{\mu t}$
	si $ \mu>0$ las soluciones crecen más que exponencialmente, si $ \mu=0$ entonces $\dot{r}=ry^2 \ge 0$ por lo que las soluciones crecen. Por lo tanto hay una bifurcación de Hopf subcrítica.

\begin{figure}[H]
 \centering
 \subfloat[$\mu<0$]{
 \includegraphics[width=0.3\textwidth]{3522ej2-1.pdf}}
 \subfloat[$\mu=0$]{
 \includegraphics[width=0.3\textwidth]{3522-2.pdf}}
 \subfloat[$\mu>0$]{
 \includegraphics[width=0.3\textwidth]{3522-3.pdf}}
\end{figure}
\end{ejemplo}

                    
