\textbf{2022-04-29}
\begin{ejemplo}
	Sea el sistema definido por
	\begin{align*}
	  \dot{x} &= y - ax \\
	  \dot{y} &= \frac{x^2}{1+x^2}-by 
	\end{align*}
	donde $x$ es una proteína y  $y$ es un transcriptor genético.

	En primer lugar vamos a calcular y graficar las ceroclinas:
	 \begin{align*}
		 \dot{x} &= 0 & y &= ax \\
		 \dot{y} &= 0 & y&=\frac{x^2}{b(1+x^2)} 
	\end{align*}

  \begin{figure}[H]
  	\centering
  	\includegraphics[width=0.5\textwidth]{020522fig1.pdf}
  	\caption{Ceroclinas del ejemplo 1}
  \end{figure}

	\textbf{Puntos de equilibrio}

	El punto de equilibrio se encuentra en la intersección de ambas ceroclinas por lo tanto:
	\begin{equation*}
	    ax = \frac{x^2}{b(1+x^2)} 
	\end{equation*}
	por lo que $x^*=0$ es un punto de equilibrio que siempre existe, mientras que si $x^*\neq 0$ entonces
	\begin{equation}
		\label{2022-05-02eq1}
		\begin{split}
			a&= \frac{x^*}{x(1+x^{{*}^2})} \\
			x^*&=ab(1+x^{{*}^2})
		\end{split}
	\end{equation}
	\begin{align*}
		abx^{{*}^2}-x^*+ab&=0 \\
		x^{{*}^2}-\frac{1}{ab}x^*+1 &= 0  
	\end{align*}
	\begin{equation}
		\label{2022-05-02eq2}
	    x^*_{1,2} = \frac{\frac{1}{ab}\pm \sqrt{\frac{1}{(ab)^2}-4}}{2} 
	\end{equation}
	podemos notar que $x_1=x_2$ si $\frac{1}{(ab)^2}-4=0$, esto quiere decir que
	\begin{align*}
	  \frac{1}{(ab)^2} &= 4 \\
		(ab)^2 &= \frac{1}{4} \\
		ab &= \frac{1}{2} \\
		a_c &= \frac{1}{2b} 
	\end{align*}
	en la bifurcación vamos a tener que $ab=\frac{1}{2}$ y volviendo a (\ref{2022-05-02eq2}) 
	\begin{align*}
	  x^*&= \frac{\frac{1}{ab}}{2} \\
	   &= \frac{1}{2ab} \\
	    &= \frac{1}{2 \frac{1}{2}} \\
	     &= 1
	\end{align*}
	Calculando las parciales para obtener la matriz jacobiana:
	\begin{align*}
		\frac{\partial f_1}{\partial x} &= -a & \frac{\partial f_1}{\partial y}  &= 1  \\
				\frac{\partial f_2}{\partial x} &= \frac{2x(1+x^2)-2xx^2}{(1+x^2)^2} & 		\frac{\partial f_2}{\partial y} &= -b  \\
		 &= \frac{2x+2x^3-2x^3}{(1+x^2)^2} \\
		 &= \frac{2x}{(1+x^2)^2} 
	\end{align*}
	por lo tanto
		\begin{equation*}
	    J = \begin{pmatrix} -a & 1 & \\ \dfrac{2x}{(1+x^2)^2} & -b \end{pmatrix}  
	\end{equation*}
	por lo que $\operatorname{tr}J = -a-b$, y dado que estamos en un modelo biológico $a>0,b>0$, por lo tanto  $\operatorname{tr}J<0$. Por otro lado
	\begin{align*}
	  \operatorname{det}J &= ab - \frac{2x}{(1+x^2)^2} 
	\end{align*}
	en el $(0,0),\ \operatorname{det} J = ab>0$ por lo tanto tenemos un punto de equilibrio estable. Ahora queremos saber cuando tenemos un punto silla, es decir cuando $ \operatorname{det} J < 0$, esto es:
	\begin{align*}
		ab - \frac{2x^*}{(1+x^{{*}^2})^2} &< 0 \\
		ab(1+x^{{*}^2})^2 - 2x^* &< 0  \\
		[ab(1+x^{{*}^2})](1+x^{{*}^2})-2x^* &< 0 
	\end{align*}
	donde de (\ref{2022-05-02eq1}) tenemos entonces
	\begin{equation*}
		\begin{split}
			x^*(1+x^{{*}^2})-2x^* &< 0 \\
			(1+x^{{*}^2})-x^* &< 0 \\
			x^{{*}^2} - 1 &< 0 \\
			x^{{*}^2} &< 1 \\
			x^* &< 1 
		\end{split}
	\end{equation*}
	dado que la bifurcación se da en $x^*=1$ y por el tipo de bifurcación tenemos que 2 puntos de equilibrio surgen a partir de 1, uno de ellos se va a la izquierda y otro a la derecha, por lo tanto el que se va a la izquierda es un punto silla, mientras que el de la derecha cumple que $x^*>1$, por lo que no cumple que el $ \operatorname{det} J <0$; por lo tanto tenemos un punto de equilibrio estable.
  
	\begin{figure}[H]
	  \begin{subfigure}[b]{0.49\textwidth}
	    \includegraphics[width=\textwidth, height=\textwidth]{020522fig2}
	  \end{subfigure}
	  \hfill
	  \begin{subfigure}[b]{0.49\textwidth}
	    \includegraphics[width=\textwidth, height=\textwidth]{020522fig3}
	  \end{subfigure}
	  \caption{Diagrama de flujo del ejemplo 1}
	\end{figure}

	\begin{tcolorbox}[colback=Black!4, colframe=White, arc=2mm]
	\begin{nota}
		Las variedades estables de punto silla siempre son separatrices.
	\end{nota}
	\end{tcolorbox}
\end{ejemplo}

\subsection{Resumen: Bifurcaciones en 2D}
\subsection*{Bifurcación transcrítica}

Forma normal
	\begin{align*}
	  \dot{x} &= \mu x - x^2 \\
	  \dot{y} &= -y  
	\end{align*}

\begin{figure}
 \centering
  \subfloat[$\mu<0$]{
   \label{f:mu<0}
    \includegraphics[width=0.3\textwidth]{020522fig4.pdf}}
  \subfloat[$ \mu=0$]{
   \label{f:mu>0}
    \includegraphics[width=0.3\textwidth]{020522fig6.pdf}}
  \subfloat[$ \mu>0$]{
   \label{f:mu>0}
    \includegraphics[width=0.3\textwidth]{020522fig7.pdf}}
		\caption{Diagrama de flujo de una bifurcación transcrítica}
 \label{f:bif trans 2d}
\end{figure}

	en la figura (\ref{f:bif trans 2d}) podemos apreciar que para $\mu<0$ tenemos dos puntos de equilibrio, donde uno es un punto silla y el otro es un nodo estable, para $ \mu=0$ los puntos chocan y para $ \mu>0$ se separan conservando su estabilidad.

	\vspace{2mm}

\begin{tcolorbox}[colback=Black!4, colframe=White, arc=0mm]                                                                       
\begin{nota}
	La creación o aniquilación de puntos de equilibrio únicamente se da en la bifurcación \textbf{nodo silla}
\end{nota}
\end{tcolorbox}                                                                                                                

\subsection*{Bifurcación tridente supercrítica}

Forma normal
\begin{align*}
  \dot{x} &= \mu x - x^3 \\
  \dot{y} &= -y 
\end{align*}
el decaimiento hacia el punto de equilibrio pasa de ser exponencial a forma de ley de potencias, conservando la estabilidad, mientras que cuando $ \mu>0$ el punto de equilibrio se transforma en un punto silla y se crean dos nodos estables.

\begin{figure}[H]
 \centering
  \subfloat[$\mu<0$]{
   \label{btp1}
    \includegraphics[width=0.3\textwidth]{020522fig8.pdf}}
  \subfloat[$\mu=0$]{
   \label{btp2}
    \includegraphics[width=0.3\textwidth]{020522fig9.pdf}}
  \subfloat[$\mu>0$]{	
   \label{btp3}
    \includegraphics[width=0.3\textwidth]{020522fig10.pdf}}
		\caption{Diagrama de flujo de la bifurcación tridente supercrítica}
 \label{biftsup}
\end{figure}

\subsection*{Bifurcación de tridente subcrítica}

Forma normal
\begin{align*}
	\dot{x} &= \mu x + x^3 \\
	\dot{y} &=  -y
\end{align*}

En este caso cuando $\mu<0$ tenemos que el $(0,0)$ es un punto de equilibrio estable y además 2 puntos de equilibrio que son del tipo nodo silla, cuando $ \mu=0$ tenemos un único punto silla donde las soluciones se acercan a $y=0$ con una velocidad de tipo ley de potencias mientras que para $ \mu>0$ tenemos un único nodo silla. 

\begin{figure}[H]
 \centering
  \subfloat[$\mu<0$]{
   \label{bts1}
    \includegraphics[width=0.3\textwidth]{020522fig11.pdf}}
  \subfloat[$\mu=0$]{
   \label{bts2}
    \includegraphics[width=0.3\textwidth]{020522fig12.pdf}}
  \subfloat[$ \mu>0$]{
   \label{bts3}
    \includegraphics[width=0.3\textwidth]{020522fig13.pdf}}
		\caption{Diagrama de flujo de la bifurcación tridente subcrítica}
\end{figure}

\begin{ejemplo}
	\begin{align*}
	 \dot{x} &= \mu x + y + \sin x \\
	 \dot{y} &= x-y 
	\end{align*}
	\textbf{Puntos de equilibrio}
	
Sabemos que $\sin 0 = 0$ por lo tanto si $x^*=0=y^*$ tenemos un punto de equilibrio, por lo tanto el $(0,0)$ es un punto de equilibrio que siempre existe.

 \textbf{¿Que tipo de bifurcación tenemos?}

 Analizando el punto de equilibrio
 \begin{align*}
	 &J = \begin{pmatrix} \mu + \cos x & 1 \\ 1 & -1 \end{pmatrix} \\
	 &\left. J \right|_{(0,0)} = \begin{pmatrix} \mu + 1 & 1 \\ 1 & -1 \end{pmatrix}   
 \end{align*}
 por lo que $$\operatorname{tr}J = 1 + \mu - 1 = \mu$$ y $$\operatorname{det} J = -\mu - 1 -1 = -\mu-2.$$
 Si $ \operatorname{det} J < 0$ tenemos un punto silla, por lo que
 \begin{equation*}
	 -\mu-2 < 0 \implies \mu>-2
 \end{equation*}
 y pasa a ser un punto de equilibrio estable en $ \mu < -2$ dado que $\operatorname{tr}J<0$. Aunque todavía no sabemos que tipo de bifurcación tenemos dado que en los tres tipos de bifurcación pasamos de tener un punto silla a un punto estable así que obtendremos los demás puntos de equilibrio.
 \vspace{2mm}
 

 \textbf{Ceroclinas}

 Recordando que nuestro sistema está definido por
\begin{equation*}
\begin{aligned}
&\dot{x}=\mu x+y+\sin x \\
&\dot{y}=x-y
\end{aligned}
\end{equation*}
tenemos entonces que 
\begin{align*}
	&y=-\mu x - \sin x \\
	&y=x
\end{align*}
\begin{figure}[H]
	\centering
	\includegraphics[width=0.5\textwidth]{020522fig14.pdf}
	\caption{Ceroclinas ejemplo 2}
\end{figure}
por lo tanto el punto de equilibrio se encuentra en la intersección, ie,
\begin{align*}
  &x=-\mu x - \sin x \\
  &0=x(1+\mu)+\sin x  
\end{align*}
haciendo una expansión en series de Taylor alrededor del cero:
\begin{align*}
  &x(1+\mu)+x - \frac{x^3}{3!} \simeq 0 \\
  &x(2+\mu)-\frac{x^3}{6} = 0  
\end{align*}
por lo que $x_1^*=0$, pero si $x^*\neq 0$
\begin{equation*}
	(2+\mu) - \frac{ x^{{*}^2}}{6} = 0 
\end{equation*}
entonces
\begin{equation*}
  x^*_{2,3} = \pm \sqrt{6(2+\mu)} 
\end{equation*}
dado que tenemos 3 puntos de equilibrio donde si $ \mu \le -2$ no existen 2 de ellos y aquel que existe es un nodo estable, mientras que si $ \mu>-2$ se forman los otros 2 nodos estables y además el $(0,0)$ cambia de estabilidad a un punto silla por lo que estamos frente a una bifurcación de tridente supercrítica.
\begin{figure}[H]
  \begin{subfigure}[b]{0.49\textwidth}
		\includegraphics[width=\textwidth, height=\textwidth]{020522fig15.pdf}
    \caption{$\mu=-3$}
  \end{subfigure}
  \hfill
  \begin{subfigure}[b]{0.49\textwidth}
    \includegraphics[width=\textwidth, height=\textwidth]{020522fig16.pdf}
    \caption{$\mu=-1$}
  \end{subfigure}
  \caption{Diagrama de flujo del ejemplo 2}
\end{figure}
\end{ejemplo}
