\chapter{Sistemas bidimensionales}
\section{Sistema lineal en 2 dimensiones}
$$
\begin{aligned}
  \dot{x}_1&=ax_1+x_2,  \\
  \dot{x}_2 &= cx_1 + dx_2 \\
\end{aligned} \ 
\begin{aligned}
    x_1(0)&=u_0 \\
  x_2(0)&=v_0  
\end{aligned}
$$

Reescribiendo nuestro sistema de manera vectorial:

$$
\begin{aligned}
  \vec{x}(t)&=
\begin{pmatrix} 
x_1(t) \\
x_2(t)
\end{pmatrix} \\
 \dot{\vec{x}} &= A \vec{x} 
 \\  \vec{x}(0) &= \vec{x_0}  \\ 
\end{aligned}
$$

donde $$
A = \begin{pmatrix} a & b \\ c & d \end{pmatrix} 
$$
$$
\vec{x}: \mathbb{R} \mapsto \mathbb{R}^2
$$
a lo que le llamamos curvas parametrizadas.
\begin{figure}[ht]
    \centering
    \incfig[0.3]{sistemalineal}
    \caption{Sistema lineal}
    \label{fig:sistemalineal}
\end{figure}

\begin{figure}[ht]
    \centering
    \incfig{curvas-parametrizadas}
    \caption{Curva parametrizada}
    \label{fig:curvas-parametrizadas}
\end{figure}
\begin{tcolorbox}[colback=Black!4,colframe=White] 
\begin{nota}[Vector velocidad]
  El vector velocidad siempre es tangente a la trayectoria y se calcula obteniendo la derivada a cada punto:
  $$
  \begin{aligned}
    \vec{v} &= \frac{d}{dt} \vec{x} \\
    &= \frac{d}{dt}(x_1,x_2) \\
    &= \left( \frac{dx_1}{dt}, \frac{dx_2}{dt} \right)  \\ 
    &= (\dot{x_1}, \dot{x_2}) \\ 
  \end{aligned}
  $$
  de esta manera
  $$
  \vec{v}=(\dot{x_1}, \dot{x_2}) = (ax_1+bx_2, cx_1+dx_2)
  $$
   estas serán las tangentes a las trayectorias, por lo que podemos reconstruir la trayectoria como las tangentes a todo $\vec{v}$.

\end{nota}
\end{tcolorbox}

\begin{ejemplo}[Resorte con una masa] Sea un resorte con una masa m.

  \begin{tcolorbox}[colback=Black!4, colframe=White]
  \begin{recordatorio}[Ley de Hooke]
  $$
  T=-kx
  $$
  \end{recordatorio}
  \end{tcolorbox}
  $$
  \begin{aligned}
    m \ddot{x} &= F \\
    m\ddot{x} &= -kx 
  \end{aligned}
  $$
  haciendo un cambio de variable:
  $$
  \begin{aligned}
    x_1 &\equiv x \\
    x_2 &\equiv \dot{x}
  \end{aligned}
  $$
  por lo que
  $$
  \dot{x_2}= \ddot{x} 
  $$
  de esta manera $m\dot{x_2}=-kx_1$.

  Organizando nuestras ecuaciones tenemos:

  $$
  \begin{aligned}
    \dot{x_1} &= x_2 \\
    \dot{x_2} &= -\frac{k}{m}x_1 \\
  \end{aligned}
  $$
  lo que implica que
  $$
   \dot{x_2}=-\omega^2x_1 , \ \text{donde } \omega^2=\frac{k}{m}  \text{ (notación) }
  $$
  $$
  \begin{aligned}
    \dot{\vec{x}}= \begin{pmatrix} 0 & 1 \\ -\omega^2 & 0 \end{pmatrix} \vec{x}  
  \end{aligned}
  $$
  Encontrando los vectores velocidad:
  $$
  \begin{aligned}
    \vec{v}=\begin{pmatrix} x_2-\omega^2x_1 \end{pmatrix}  
  \end{aligned}
  $$

\begin{figure}[ht]
    \centering
    \incfig[0.5]{vectoresvel}
    \caption{Vectores velocidad $\vec{v}=(x_2,-\omega^2x_1)$ para los puntos $(0,1),(1,0),(0,-1),(-1,0)$}
    \label{fig:vectoresvel}
\end{figure}

\begin{figure}[!tbp]
  \begin{subfigure}[b]{0.49\textwidth}
    \includegraphics[width=\textwidth, height=\textwidth]{sb1}
    \caption{Campo vectorial de velocidades con $\omega^2=2$}
  \end{subfigure}
  \hfill
  \begin{subfigure}[b]{0.49\textwidth}
    \includegraphics[width=\textwidth, height=\textwidth]{sb2}
    \caption{Trayectorias de las soluciones.}
  \end{subfigure}
\end{figure}

\begin{figure}[ht]
    \centering
    \incfig[0.4]{proyeccionx2}
    \caption{Proyección de $x_2$ respecto a alguna curva cerrada del ejemplo 1.}
    \label{fig:proyeccionx2}
\end{figure}


notemos que en $(0,0)$ es un punto de equilibrio, pues la velocidad es cero.

 \begin{tcolorbox}[colback=Black!4,colframe=White] 
\begin{nota}

  $(0,0)$ es un punto de equilibrio para cualquier punto lineal.
  
\end{nota}
\end{tcolorbox}

\end{ejemplo}

\begin{ejemplo} Consideremos el siguiente sistema lineal $$
  \dot{\vec{x}} = \begin{pmatrix} a & 0 \\ 0 & -1 \end{pmatrix} \vec{x}
$$
con $a \in \mathbb{R}$ y $\vec{x}=\begin{pmatrix} x_1 \\ x_2 \end{pmatrix} $.

Reescribiendo nuestro sistema:

$$
\begin{aligned}
  \dot{x_1} &= ax_1 \\ 
  \dot{x_2} &= -x_2 \\ 
\end{aligned}
$$
es un sistema desacoplado, porque para la ecuación de $x_1$ no le importa la solución de $x_2$ y viceversa.

Soluciones:

$$
\begin{aligned}
  x_1(t) &= k_1e^{at} \\
  x_2(t) &= k_2e^{-t} \\ 
\end{aligned}
$$
  si tenemos $\vec{x}(0)=\vec{x_0}$, entonces

  $$
  \vec{x}(0)=\begin{pmatrix} x_1(0) \\ x_2(0) \end{pmatrix} = \begin{pmatrix} k_1 \\ k_2 \end{pmatrix} = \vec{x_0}
  $$

  Caso I: $a<-1$. Si $a<-1$ entonces  $x_1(t)$ decrece más rápido que $x_2(t)$, por lo que se va a pegar a cero mucho más rápido, y ambas va a tender a $(0,0)$. Notemos que cuando $\lim_{t \to \infty} $ las trayectorias son paralelas a $x_2$ mientras que cuando $\lim_{n \to -\infty} $ las trayectorias son paralelas a $x_1$.

  \begin{tcolorbox}[colback=Black!4,colframe=White] 
  \begin{nota}[Nodo estable]
    Al $(0,0)$ le llamaremos nodo estable porque no importa si empezamos un poco alejado de el, siempre vamos a terminar en el  $(0,0)$
  \end{nota}
\end{tcolorbox}

  Caso II:  $a=-1$. En este caso tendremos que $x_1(t)=x_2(t)$, por lo que nuestras trayectorias serán rectas. Tendremos un nodo estrella estable pues no hay ninguna deformación en las trayectorias.
   \vspace{2mm}
  
  Caso III: $-1<a<0$. Tendremos el caso contrario al caso I pues ahora  $x_1(t)$ será menos decreciente que $x_2(t)$.
  \vspace{2mm}
  
  Caso IV: $a=0$. En este caso  $x_1(t)=k_1$ por lo que su valor depende únicamente de la condición inicial, mientras que  $x_2(t)$ si va a decrecer exponencialmente.  Notemos que todo el eje $x$ es una linea de puntos de equilibrio estable.
\vspace{2mm}

  Caso V:  $a>0$. En este caso $x_1(t)$ va a tender a infinito mientras que $x_2(t)$ va a tender a cero. No importa donde empezamos siempre nos vamos a acercar hacia el eje $x$ por lo mismo que  $x_2(t)$ tiende a cero. En este caso tenemos 2 curvas importantes. Si empezamos con una condición inicial $x_1(0)=0$ entonces $x_1(t)=0$ por lo que no jugaría ningún papel, yéndonos a $(0,0)$, a esa linea que nos llevan al  $(0,0)$ se llama  \textbf{la variedad estable}, mientras que al eje $x$ se le llama  \textbf{la variedad inestable} (no hay nada que nos atraiga, todo nos repele). A esta geometría se le llama \textbf{punto silla}. Y en general el punto silla siempre es inestable.                 

  \begin{figure}[!tbp]
    \begin{subfigure}[b]{0.4\textwidth}
      \includegraphics[width=\textwidth, height=\textwidth]{sb3}
      \caption{$a<-1$}
    \end{subfigure}
    \hfill
    \begin{subfigure}[b]{0.4\textwidth}
      \includegraphics[width=\textwidth, height=\textwidth]{sb4}
      \caption{$a=-1$}
    \end{subfigure}
    \begin{subfigure}[b]{0.4\textwidth}
      \includegraphics[width=\textwidth, height=\textwidth]{sb5}
      \caption{$-1<a<0$}
    \end{subfigure} \hfill 
     \begin{subfigure}[b]{0.4\textwidth}
      \includegraphics[width=\textwidth, height=\textwidth]{sb6}
      \caption{$a=0$}
    \end{subfigure}
    \begin{subfigure}{0.4\textwidth}
      \includegraphics[width=\textwidth, height=\textwidth]{sb7}
      \caption{$a>0$}
    \end{subfigure}
  \end{figure}
\end{ejemplo}


