\section{Ecuaciones en diferencias no lineales}

\begin{comment}
	\begin{table}[H]
		\centering
		\caption{caption}
		\label{tab:label}
	
		\begin{tabular}{c}
			Ecuaciones en diferenciales
			\begin{itemize}
				\item Puntos de equilibrio
							\begin{equation*}
							  \dot{x}=0 
							\end{equation*}
				\item No tenemos oscilaciones en 1D
				\item $x^*$ es estable si $r<0$
				 \item $x^*$ es inestable si $r>0$
	
			\end{itemize}
		\end{tabular}
	\end{table}
\end{comment}

Una ecuación en diferencias no lineal se puede escribir como $x_{n+1}=f(x_n)$ con $f$ no lineal. Recordando que los puntos de equilibrio son soluciones constantes:
 \begin{align*}
  x_n &= x^* \ \forall n \\
	x_{n+1}&=x^*
\end{align*}
por lo que
\begin{equation*}
  x^*=f(x^*) 
\end{equation*}
podemos encontrar los puntos de equilibrio en la intersección entre la identidad y $f(x)$

\textbf{Estabilidad}

Sea $x_{n} \equiv x^* + \tilde{x}_n$, donde $x^*$ es un punto de equilibrio y $\tilde{x}_n$ es una desviación pequeña, sustituyendo
\begin{equation*}
  x^* + \tilde{x}_{n+1} = f(x^*+\tilde{x}_n) 
\end{equation*}
	haciendo una aproximación en series de Taylor alrededor de $x^*$
\begin{align*}
												& \simeq f(x^*) + \tilde{x}_n \left. \frac{df}{dx} \right|_{x^*} \quad \lambda \equiv \left. \frac{df}{dx} \right|_{x^*} \\
												&= f(x^*) + \lambda \tilde{x}_n \\
												&= x^* + \lambda \tilde{x}_n
\end{align*}
por lo tanto
\begin{equation*}
  \tilde{x}_{n+1} = \lambda \tilde{x}_n 
\end{equation*}
Entonces:

\begin{itemize}
	\item \begin{equation*}
  \left| \left. \frac{df}{dx} \right|_{x^*}  \right| < 1 \implies x^* \text{ es estable } 
\end{equation*} 
\item \begin{equation*}
  \left| \left. \frac{df}{dx} \right|_{x^*}  \right|  >1 \implies x^* \text{ es inestable }
\end{equation*}
\item \begin{equation*}
			  \left. \frac{df}{dx} \right|_{x^*}<0 \implies \text{ las soluciones oscilan cerca de $x^*$ }  
			\end{equation*} 
\item \begin{equation*}
			   \left. \frac{df}{dx} \right|_{x^*}  = 1 \text{ no podemos determinar la estabilidad de esta forma } 
			\end{equation*}
\end{itemize}

\begin{ejemplo}
	\begin{equation*}
	  x_{n+1}=x_n^2 
	\end{equation*}
	\textcolor{blue}{insertar gráfica}
	Estabilidad de los puntos de equilibrio

	\begin{align*}
	  \frac{df}{dx} &= 2x \\
		\left. \frac{df}{dx} \right|_{x^*=0} &= 0
	\end{align*}
	por lo tanto $x^*=0$ es un punto de equilibrio estable, por otro lado
	\begin{align*}
		\left. 	  \frac{df}{dx}  \right|_{x^*=1} &= 2 
	\end{align*}
	por lo tanto $x^*=1$ es inestable.

	Método gráfico para comprobar la estabilidad de los puntos de equilibrio: \textbf{Cobwebbing}

	
\end{ejemplo}

\begin{ejemplo}
	\begin{equation*}
	  x_{n+1}=\cos(x_n),\ x^*=cos(x^*)=0.7 
	\end{equation*}

	\begin{equation*}
	  \begin{split}
	  	\frac{df}{dx} &= -\sin x \\
			\left. \frac{df}{dx} \right|_{0.7} &\approx -0.64  
	  \end{split} 
	\end{equation*}
	por lo tanto tenemos un punto de equilibrio estable oscilatorio 

\end{ejemplo}


