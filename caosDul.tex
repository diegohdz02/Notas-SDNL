\section{Caos y atractores extraños}

\begin{comment}
	Cuando tenemos un atractor extraño tenemos caos.
	
\end{comment}

\textbf{Ecuaciones de Hénon}
\begin{align*}
  x_{n+1} &= y_n+1-ax_{n}^2 \\
	y_{n+1}&= bx_n
\end{align*}
con $a,b$ parámetros y $0<b<1$.

Eligiendo $b=0.3$ y $a=1.4$ 

\begin{tcolorbox}[colback=Black!4, colframe=White,arc=2mm]
\begin{definicion}[Atractor]
	Conjunto mínimo $A$ tal que soluciones en cierta región del espacio fase se acercan al conjunto cuando $t,n \to \infty$. 
\end{definicion}
\end{tcolorbox}

Mecanismo general de los atractores extraños
\begin{enumerate}
	\item Acercar las soluciones al atractor (compresión)
	\item Extender las soluciones (expansión)
	\item Doblamiento de las soluciones sobre si mismas (compresión)
\end{enumerate}

Sistema tridimensional más sencillo con soluciones caóticas.

\textbf{Sistema de Rossler}
\begin{align*}
  \dot{x} &= -y-z \\
	\dot{y} &= x+ay \\
	\dot{z}= b+z(x-c)
\end{align*}
con $a,b,c$ parámetros. Las parámetros del sistema anterior que se aprecian mejor son  $a=b=0.2$ variando  $c$.
 
Condiciones que cumplen los máximos: $\dot{x}=0 \implies y(t)=-z(t)$, y para asegurnos que estamos tomando un máximo entonces $\ddot{x} <0$:
\begin{align*}
  \ddot{x} &= -\dot{y} - \dot{z} \\
					 &= -x-ay-b-z(x-c)<0,
\end{align*}
variando $c$.

 \section{Extracción del atractor}
 \begin{enumerate}
 	\item Observaciones $x(t)$ 
	\item Generamos una trayectoria tridimensional:
		\begin{align*}
		  \vec{y}(t) &= \left( x(t), x(t+\tau), x(t+\tau) \right)
		\end{align*}
		con $ \tau$ constante.
	\item Si esto converge a un atractor extraño entonces mi sistema experimental es caótico. 
 \end{enumerate}



