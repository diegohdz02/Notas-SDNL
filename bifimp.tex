\section{Bifurcaciones imperfectas}

\begin{tcolorbox}[colback=Black!4, colframe=White]
\begin{recordatorio}
    Bifurcaciones de tridente supercrítica  
    $$
  \dot{x}= rx-x^{3}   
    $$
    $$
    \dot{x}=h+rx-x^{3}
    $$
    si $h>0$ entonces  $\dot{x}$ será un poco más positivo, ie, las soluciones van a crecer un poco más, análogamente con $h<0$ habrá una preferencias hacia soluciones negativas. De esta manera diremos que  $h$ es un parámetro de imperfección, rompe la simetría de la ecuación.
\end{recordatorio}
\end{tcolorbox}
 \begin{figure}[ht]
    \centering
      \incfig[0.4]{bifh1}
      \subcaption{$h<0$ }
      \label{fig:bifh1}
    \incfig[0.4]{bifh2}
      \subcaption{$h>0$}
      \label{fig:bifh2}
    \caption{Diagrama de flujo del sistema $\dot{x}=h+rx-x^3$}
\end{figure}  

Puntos de equilibrio

$$
\begin{aligned}
  0 &= h+r\hat{x}-\hat{x}^3 \\
\end{aligned}
$$                            
resolver la ecuación anterior puede ser complicado, por lo que vamos a definir
$$
-h = rx-x^3
$$
y hacemos $$
f(x)=-h , \ g(x)=rx-x^3  
$$
por lo tanto las soluciones de la ecuación estará dado por los puntos $f(x)=g(x)$. \

\begin{figure}[ht]
    \centering
    \incfig[0.4]{bi1}
    \subcaption{$r=2,\ h=-1.11 $ }
    \label{fig:bifurcación1}
    \incfig[0.4]{bi2}
    \subcaption{$r=2, \ h=0$}
    \label{fig:bifurcación2}
    \incfig[0.4]{bi3}
    \subcaption{$r=2, \ h=1.11$}
\end{figure}



Podemos apreciar que únicamente habrá un punto de equilibrio para $r<0$ y  $r=0$, mientras que para  $r>0$ para ciertas  $h$ podemos tener de 1 a 3 puntos de equilibrio. Para $r=2$ y  $h=-1.11$ tenemos una bifurcación de tipo nodo-silla, pues \textbf{hay dos puntos de equilibrio que aparecen de la nada} y para $r=2$ y  $h=1.11$ volvemos a tener una bifurcación de nodo silla pues  \textbf{dos puntos de equilibrio se encuentran y se aniquilan.}

$$
\begin{aligned}
  \frac{d}{dx}g(x) &= 0 \\ 
  r-3x^2 &= 0 \\ 
  3x^2 &= r  \\
  x_c &= \pm \sqrt{\frac{r}{3}} \\ 
\end{aligned}
$$
Evaluando:

$$
\begin{aligned}
  g(x_c)&=rx_c - x_c^3  \\
	&= r \left( \pm \sqrt{\frac{r}{3}} \right) - \left( \pm \sqrt{\frac{r}{3}} \right)^2 \left( \pm \sqrt{\frac{r}{3}} \right) \\
  &= \pm r \sqrt{\frac{r}{3}} - \frac{r}{3} \left( \pm \sqrt{\frac{r}{3}} \right)  \\ 
  &= \pm r \sqrt{\frac{r}{3}} \mp \frac{r}{3} \sqrt{\frac{r}{3}} \\
  &= \sqrt{\frac{r}{3}} \left( \pm r \mp \frac{r}{3} \right)  \\
  &= \pm \frac{2r}{3} \sqrt{\frac{r}{3}} \\
  &= \pm h_c \\ 
\end{aligned}
$$
de esta manera ya tenemos una ecuación que me representa a ambos parámetros, graficando $h_c$:

\begin{figure}[ht]
    \centering
    \incfig[0.4]{pc1}
    \caption{Diagrama de estabilidad}
    \label{fig:pc1}
\end{figure}

En el diagrama de estabilidad podemos apreciar la cúspide, donde los puntos de equilibrio se encuentran y las curvas definidas por $h_c = \pm \frac{2r}{3} \sqrt{\frac{r}{3}}$. En la parte amarilla del diagrama tenemos únicamente un punto de equilibrio mientras que en la azul tenemos 3.  

\begin{tcolorbox}[colback=Black!4,colframe=White] 
  \begin{nota}[Espacio de parámetros y diagrama de estabilidad]
  Al espacio $h$ vs $r$, se le llama \textbf{espacio de parámetros} y lo que graficamos sobre el espacio se le llama \textbf{diagrama de estabilidad. } 
\end{nota}
\end{tcolorbox}

\begin{figure}[ht]
    \centering
    \incfig[0.4]{4bif1}
    \caption{Diagrama de bifurcación fijando $h=0.01$}
    \label{fig:4bif1}
\end{figure}

\begin{figure}[ht]
    \centering
    \incfig[0.4]{4bif2}
    \caption{Diagrama de bifurcación fijando $r=1$}
\end{figure}    

\begin{figure}[H]
    \centering
    \incfig[0.45]{supcat}
    \caption{Superficie de catástrofe $h+r\hat{x}-\hat{x}^3$}
    \label{fig:supcat}
\end{figure}   
 

